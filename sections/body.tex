\section*{Introduction}
Shen's personal lecture notes for the Goldsmiths 2019 --- 2020 CM1015 Numerical Mathematics course. Please contact Shen Zhou Hong at \href{mailto:sh333@student.london.ac.uk} for any corrections or editorial comments.

\section*{Cartesian Mathematics}

\subsection*{Defining the Domain and Range of a Function}
The domain (set of valid inputs) and range (set of valid outputs) of a function can be defined using an variety of different notation, such as set builder notation, interval notation, and inequalities.

\subsubsection*{Set-Builder Notation}
\begin{align*}
  \left\{x\ |\ x > 0\right\}
\end{align*}
\noindent
This is read as: "the set of all x's, such that x is greater than zero".

\subsubsection*{Interval Notation}
Interval notation uses square and round brackets. Square $[]$ brackets indicate we include those end values, while round brackets $()$ indicate we don't.

\begin{align*}
  \left[0, 20\right]
\end{align*}
\noindent
This denotes the interval of numbers from zero to twenty, including both zero and twenty.

\begin{align*}
  \left[1, 5\right)
\end{align*}
\noindent
This denotes the interval of numbers from one until but not including 5, i.e. "1, 2, 3, 4".
